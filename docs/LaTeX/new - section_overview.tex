\section{Overview and Objectives}
\begin{itemize}
  \item \textbf{Problem.} Metals emit weakly in equilibrium (essentially blackbody). Under monochromatic pumping, non-equilibrium electron and hole populations can enhance emission, potentially revealing interband signatures even in noble metals dominated by Drude-like intraband response.
  \item \textbf{Goal.} Quantify the interband contribution to non-equilibrium emission from gold nanoparticles and compare it to the well-understood intraband background.
  \item \textbf{Strategy.} (i) Reproduce known intraband emission in energy and momentum space to validate numerics and approximations. (ii) Extend the same formalism to interband transitions near the X and L valleys, where anisotropy requires $k$-space treatment. (iii) Combine both channels into a complete spectrum under pumping.
  \item \textbf{Thesis outputs to date.} Complete intraband derivations (equilibrium and non-equilibrium), numerical validation of Gaussian delta approximations, identification and correction of distribution-function errors in the literature, a suite of reproducible comparison plots, and a working plan plus tooling for the anisotropic interband calculation.
\end{itemize}
