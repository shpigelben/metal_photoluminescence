\section{Numerical Methodology}
\subsection{Discretization choices}
\begin{itemize}
  \item Energy grid: $E\in[0,12]$~eV with $\mathcal{O}(10^{3})$ points; sufficient to resolve the Fermi edge and photon energies of interest.
  \item Momentum grid: cylindrical coordinates $(k_{\perp},k_{\parallel})$ with Gaussian delta enforcement; grid sizes tuned via the convergence rules in Section~\ref{sec:delta}.
  \item Delta approximation: Gaussian of width $\sigma$ with window $\pm m\sigma$ and step $\Delta x=s\sigma$ to balance approximation and quadrature error.
  \item Units: plots are normalized to emphasize spectral shape; absolute prefactors ($\mu$, $V$, photonic density) can be reinstated for experimental comparison.
\end{itemize}

\subsection{Software stack}
\begin{itemize}
  \item Python (NumPy/Matplotlib) for the comparison plots in Section~\ref{sec:results}.
  \item Jupyter notebooks in \texttt{code/} for full emission calculations (energy and momentum space; isotropic and anisotropic cases).
  \item \texttt{.venv} virtual environment (local) with only NumPy/Matplotlib to keep reproducibility and avoid system pollution.
\end{itemize}

\subsection{Convergence strategy for Gaussian deltas}\label{sec:delta}
Using the guidelines from the ``Numeric Delta Approximation'' note:
\begin{itemize}
  \item Fix $m$ (window multiplier, e.g., $m=10$) and $s$ (step fraction, e.g., $s=0.1$); then $N\geq \lceil 2m/s\rceil$ samples resolve the peak.
  \item Reduce $\sigma$ until the relative change in the integral is below a target tolerance (e.g., $10^{-3}$), compensating with a finer grid as needed.
  \item For anisotropic valleys, transform to $(q_{\perp},q_{\parallel})$ first so that the Gaussian depends only on $q^{2}$, simplifying convergence.
\end{itemize}
