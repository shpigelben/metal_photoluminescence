\section{Physical Background}
\subsection{Photoluminescence channels in metals}
Electronic emission can arise from intra- or interband transitions:
\begin{itemize}
  \item \textbf{Intraband} ($c\to c$): dominant in equilibrium for Au; momentum change requires phonon assistance or relaxation of strict momentum conservation in a model.
  \item \textbf{Interband} ($c\leftrightarrow v$): negligible in equilibrium for Au because $5d$ holes are absent, but under pumping with $\hbarwl$ above the threshold, hole populations in $5d$ enable radiative recombination.
  \item Total electronic contribution: $\Gamma=\Gamma_{cc}+6\Gamma^{X}_{cv}+8\Gamma^{L}_{cv}$ (fcc degeneracies).
\end{itemize}

\subsection{Energy and momentum conservation}
\begin{itemize}
  \item Direct one-step intraband transitions are momentum-forbidden because $\mathbf{q}_{\gamma}\ll \mathbf{k}_{e}$. Models that yield emission either (a) embed phonon assistance implicitly or (b) relax strict momentum conservation when integrating over $k$.
  \item Direct interband transitions near symmetry points can conserve momentum because the initial and final states are in different bands at the same crystal momentum.
  \item These constraints motivate explicit treatment in $k$ space for interband work.
\end{itemize}

\subsection{Gold band structure (working model)}
\begin{itemize}
  \item Conduction band: nearly parabolic, isotropic with effective mass close to $m_{e}$; dispersion $\E_{c}=\hbar^{2}k^{2}/2m_{e}$.
  \item Valence band pockets near X and L: anisotropic ellipsoids with effective masses $(m_{b\perp},m_{b\parallel})$ and offsets $\E_{0b}$ from literature on Au nanowires and nonlinear SPP dynamics.
  \item Degeneracy: $6$ X points, $8$ L points; these factors multiply valley-specific rates.
\end{itemize}
