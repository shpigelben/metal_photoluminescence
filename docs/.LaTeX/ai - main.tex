\documentclass[11pt]{article}

\usepackage[margin=1in]{geometry}
\usepackage{amsmath,amssymb,bm}
\usepackage{siunitx}
\usepackage{physics}
\usepackage{enumitem}
\usepackage{hyperref}
\usepackage{graphicx}
\usepackage{cite}

\graphicspath{{ai - figs/}}

\hypersetup{
  colorlinks=true,
  linkcolor=blue,
  urlcolor=blue,
  citecolor=blue
}

\setlist[itemize]{topsep=3pt,itemsep=3pt,leftmargin=14pt}
\setlist[enumerate]{topsep=3pt,itemsep=3pt,leftmargin=16pt}

\newcommand{\E}{\mathcal{E}}
\newcommand{\feq}{f^{\mathrm{T}}}
\newcommand{\fne}{f}
\newcommand{\Edos}{\rho}
\newcommand{\hbarw}{\hbar\omega}
\newcommand{\hbarwl}{\hbar\omega_{\mathrm{L}}}
\newcommand{\kB}{k_{\mathrm{B}}}
\newcommand{\EF}{\mathcal{E}_{\mathrm{F}}}
\newcommand{\BE}{f_{\mathrm{BE}}}
\newcommand{\FD}{f_{\mathrm{FD}}}

\title{Contribution of Interband Transitions to Non-equilibrium Emission from Gold\\[0.5em]Thesis Progress Document}
\author{}
\date{\today}

\begin{document}

\maketitle

\begin{center}
  \includegraphics[width=0.25\textwidth]{ai - figs/bgu_logo.png}
\end{center}

\begin{abstract}
This document consolidates the thesis work completed so far on non-equilibrium photoluminescence from gold. It follows the structure and notation used in the working summary \textit{Contribution of Interband Transitions to Non-equilibrium Emission From Gold}, expanding each element into a thesis-length narrative. The report reviews the physical background, lays out the emission formalism in both momentum and energy space, records the derivations and numerical benchmarks achieved, documents corrections to prior literature, and sets a clear path to finalize the interband calculations in anisotropic bands. All content is self-contained for Overleaf compilation.
\end{abstract}

\tableofcontents
\newpage

\input{ai - section_overview.tex}
\section{Literature Context and Critique}
\subsection{Key references}
\begin{itemize}
  \item Drude intraband emission under steady pumping (Sivan \& Dubi)---provides the energy-space benchmark reproduced here.
  \item ``Theory of Hot Photoluminescence from Drude Metals''---useful analytic series but uses a non-thermal distribution that overpopulates states; corrected below.
  \item Gold band-structure parameters near X/L (nonlinear SPP dynamics in Au nanowires; Rosei)---supply effective masses and offsets for anisotropic valleys.
  \item Photocatalysis literature for relaxation-time non-equilibrium distributions---basis for the corrected $f(\E;\omega_{\mathrm{L}})$ used here.
\end{itemize}

\subsection{Findings from the review}
\begin{itemize}
  \item \textbf{Distribution correctness.} Prior Drude-only derivations insert a non-thermal correction that violates Pauli exclusion. Using the Boltzmann+relaxation-time form in Eq.~\eqref{eq:nonthermal_dist} fixes this.
  \item \textbf{Density-of-states handling.} Several sources collapse $\rho_{J}$ into a single factor; keeping both initial and final eDOS terms is essential whenever $\hbarw$ approaches or exceeds $\EF$.
  \item \textbf{Momentum conservation.} Experimental agreement of intraband spectra with models that ignore momentum conservation implies an implicit phonon channel; for interband, direct transitions are available and should be treated explicitly.
\end{itemize}

\input{ai - section_background.tex}
\section{Emission Formalism}
\subsection{General $k$-space expression}
Starting from Fermi's golden rule, the spontaneous emission rate for band $\alpha\to\beta$ transitions is
\begin{equation}
  \Gamma_{\alpha\beta}(\hbarw)=\frac{2\pi}{\hbar}\!\left(\frac{2V}{(2\pi)^{3}}\right)^{\!2}\!\iint_{\mathrm{BZ}}\!|\mu_{\alpha\beta}(\mathbf{k}_{\mathrm{i}},\mathbf{k}_{\mathrm{f}})|^{2}\,f_{\alpha}(\mathbf{k}_{\mathrm{i}})\bigl[1-f_{\beta}(\mathbf{k}_{\mathrm{f}})\bigr]\delta\!\left(\E_{\beta}(\mathbf{k}_{\mathrm{f}})-\E_{\alpha}(\mathbf{k}_{\mathrm{i}})-\hbarw\right)\,d^{3}k_{\mathrm{i}}\,d^{3}k_{\mathrm{f}}.
  \label{eq:general_rate}
\end{equation}
The transition dipole $\mu$ encodes selection rules and, implicitly, momentum transfer.

\subsection{Energy-space reduction for isotropic bands}
For isotropic dispersions and constant dipole moment $|\mu_{\alpha\beta}|\to \mu_{\alpha\beta}$, the density of states
\begin{equation}
  \Edos(\E)=\frac{4\pi m^{3/2}}{\hbar^{3}}\sqrt{2\E}
  \label{eq:dos}
\end{equation}
allows reduction of Eq.~\eqref{eq:general_rate} to
\begin{equation}
  \Gamma_{\alpha\beta}(\hbarw)=\frac{2\pi}{\hbar}\!\left(\frac{2\mu_{\alpha\beta}V}{(2\pi)^{3}}\right)^{\!2}\!\!\int f_{\alpha}(\E)\bigl[1-f_{\beta}(\E-\hbarw)\bigr]\Edos_{\alpha}(\E)\Edos_{\beta}(\E-\hbarw)\,d\E.
  \label{eq:energy_rate}
\end{equation}
This is the reference form used to validate the momentum-space numerics.

\subsection{Joint density factor}
For clarity, the joint factor entering Eq.~\eqref{eq:energy_rate} is
\begin{equation}
  \rho_{J}(\E_{\mathrm{i}},\E_{\mathrm{f}})=f(\E_{\mathrm{i}})\Edos(\E_{\mathrm{i}})\bigl[1-f(\E_{\mathrm{f}})\bigr]\Edos(\E_{\mathrm{f}}),
\end{equation}
maintaining both initial and final densities explicitly (correcting a misuse in the referenced literature).

\section{Supporting Derivations}
\subsection{Electronic density of states (A2)}
For an isotropic parabolic band $\E(k)=\hbar^{2}k^{2}/2m$,
\begin{equation}
  d^{3}k\to4\pi k^{2}dk=4\pi k^{2}\frac{dk}{d\E}d\E=\frac{4\pi m^{3/2}}{\hbar^{3}}\sqrt{2\E}\,d\E,
\end{equation}
yielding Eq.~\eqref{eq:dos}. This mapping is reused when isotropizing anisotropic valleys.

\subsection{Change of variables for anisotropic ellipsoids}
For $\E_{b}=\E_{0b}\pm \hbar^{2}(k_{\perp}^{2}/2m_{b\perp}+k_{\parallel}^{2}/2m_{b\parallel})$ define
\begin{equation}
  k_{\perp}=\frac{q_{\perp}}{\sqrt{m_{b\perp}}},\qquad k_{\parallel}=\frac{q_{\parallel}}{\sqrt{m_{b\parallel}}},
  \label{eq:anisotropic_cov}
\end{equation}
so that $\E_{b}=\E_{0b}\pm \hbar^{2}(q_{\perp}^{2}+q_{\parallel}^{2})/2$ and $d^{3}k=\frac{1}{\sqrt{m_{b\perp}m_{b\parallel}}}\,2\pi q_{\perp}\,dq_{\perp}\,dq_{\parallel}$. The energy-conservation delta then depends only on $q^{2}$, reducing interband valley integrals to coupled 1D forms akin to Eq.~\eqref{eq:energy_rate}.

\subsection{Delta-function approximation}
Replace $\delta(x)$ by $G_{\sigma}(x)=\bigl(\sigma\sqrt{2\pi}\bigr)^{-1}\exp[-x^{2}/(2\sigma^{2})]$ with joint limit $\sigma\to0$ and grid size $N\to\infty$. For trapezoidal quadrature the numeric error scales as $N^{-2}$, while approximation error shrinks with $\sigma$. Practical rule used in notebooks: integrate over $\pm m\sigma$ with step $\Delta x=s\sigma$ (e.g., $m=10$, $s\leq 0.1$) so $N\geq \lceil2m/s\rceil$ samples resolve the peak without flattening normalization. In 2D, maintain the Jacobian of the transformed coordinates.

\subsection{Quantum Boltzmann equation and non-equilibrium distribution}
At steady state,
\begin{equation}
  \left(\frac{\partial f}{\partial t}\right)_{\mathrm{ex}}+\left(\frac{\partial f}{\partial t}\right)_{e\text{--}e}+\left(\frac{\partial f}{\partial t}\right)_{e\text{--}ph}\approx 0,
\end{equation}
with excitation rate from Fermi's golden rule and $e$-$e$ collisions treated via a relaxation time $\tau_{e\text{--}e}$. Neglecting $e$-$ph$ far from $\EF$ yields
\begin{equation}
  f(\E;\omega_{\mathrm{L}})=\feq(\E)+\delta_{E}(\omega_{\mathrm{L}})\,B(\E;\omega_{\mathrm{L}}),
  \label{eq:nonthermal_dist}
\end{equation}
with
\begin{align}
  B(\E;\omega_{\mathrm{L}})&=\feq(\E-\hbarwl)\bigl[1-\feq(\E)\bigr]-\feq(\E)\bigl[1-\feq(\E+\hbarwl)\bigr], \label{eq:B_kernel}\\
  \delta_{E}(\omega_{\mathrm{L}})&=\left|\tau_{e\text{--}e}R(\omega_{\mathrm{L}})E_{\mathrm{L}}\right|^{2},\qquad R(\omega_{\mathrm{L}})=\frac{4\epsilon_{0}\epsilon_{m}''(\omega_{\mathrm{L}})}{3\hbar n_{e}}\frac{\EF}{\hbarwl}. \label{eq:deltaE}
\end{align}
This form avoids unphysical occupations $>1$ present in the earlier Drude-only expression.

\input{ai - section_numerical.tex}
\input{ai - section_intraband.tex}
\section{Results and Comparisons}\label{sec:results}
\subsection{Equilibrium intraband benchmarks}
Figure~\ref{fig:intra-eq} shows numeric spectra with and without the $\E$-dependent eDOS compared to the analytic constant-eDOS shape. The energy-dependent eDOS slightly hardens the spectrum at larger $\hbarw$, consistent with the increasing $\sqrt{\E}$ weight.

\begin{figure}[h]
  \centering
  \includegraphics[width=0.7\textwidth]{ai - figs/intraband_equilibrium.png}
  \caption{Equilibrium intraband emission: numeric (with/without eDOS) vs. analytic constant-eDOS shape. All curves normalized to unit peak.}
  \label{fig:intra-eq}
\end{figure}

\subsection{Non-equilibrium enhancement}
Figure~\ref{fig:intra-noneq} illustrates non-equilibrium spectra for two pump strengths ($\delta_{E}=0.02,0.05$) at $\hbar\omega_{\mathrm{L}}=2$~eV. The enhancement is stronger near the Fermi edge where $f(1-f)$ is largest; spectra smoothly revert to equilibrium as $\delta_{E}\to0$.

\begin{figure}[h]
  \centering
  \includegraphics[width=0.7\textwidth]{ai - figs/intraband_noneq.png}
  \caption{Non-equilibrium intraband emission normalized to peak. Dashed line: equilibrium baseline.}
  \label{fig:intra-noneq}
\end{figure}

\subsection{Gaussian delta convergence}
Figure~\ref{fig:delta} uses a toy integrand to demonstrate convergence of the Gaussian delta approximation; relative error drops rapidly as $\sigma$ shrinks, confirming the sampling rule in Section~\ref{sec:delta}. The same behavior is observed in the notebook tests for the physical emission integrals.

\begin{figure}[h]
  \centering
  \includegraphics[width=0.6\textwidth]{ai - figs/delta_convergence.png}
  \caption{Convergence of Gaussian delta approximation vs. width $\sigma$ (log scale).}
  \label{fig:delta}
\end{figure}

\subsection{Momentum conservation discussion}
\begin{itemize}
  \item One-step intraband transitions are momentum-forbidden; agreement with experiment implies effective inclusion of phonon assistance or relaxed momentum constraints.
  \item Future refinement could add an explicit phonon spectral function to enforce momentum conservation without suppressing emission.
\end{itemize}

\input{ai - section_detailed_intra.tex}
\section{Parameter Sensitivity and Physical Trends}
\subsection{Temperature dependence}
\begin{itemize}
  \item Higher $T$ broadens $\FD$, boosting low-$\hbarw$ emission and reducing sharpness near $\EF$.
  \item Analytic Eq.~\eqref{eq:analytic_eq} captures this via the Bose factor; energy-dependent $\Edos$ adds a mild hardening at larger $\hbarw$.
\end{itemize}

\subsection{Pump photon energy $\hbarwl$}
\begin{itemize}
  \item Non-equilibrium enhancement scales with $B(\E;\omega_{\mathrm{L}})$; as $\hbarwl$ grows, the terms in Eq.~\eqref{eq:B_kernel} separate and $\delta f$ grows until phase-space closes.
  \item For interband work, $\hbarwl$ must exceed the $5d\to6sp$ threshold; below threshold $\Gamma^{\nu}_{cv}$ vanishes.
\end{itemize}

\subsection{Pump intensity (through $\delta_{E}$)}
\begin{itemize}
  \item Small $\delta_{E}$: linear regime governed by $A(\hbarw)$; spectra shift modestly.
  \item Moderate $\delta_{E}$: quadratic corrections $B(\hbarw)\delta_{E}^{2}$ appear; clipping keeps occupations physical.
  \item Large $\delta_{E}$: would require adding recombination and $e$-$ph$ cooling; to be addressed if experimental conditions demand it.
\end{itemize}

\subsection{Effect of the density of states}
\begin{itemize}
  \item $\Edos(\E)\propto\sqrt{\E}$ enhances high-energy emission by weighting higher initial/final states.
  \item For anisotropic valleys, effective masses reshape $\Edos$; isotropization via Eq.~\eqref{eq:anisotropic_cov} provides an analytic route to compute the modified DOS.
\end{itemize}

\input{ai - section_experimental.tex}
\section{Thesis Structure Outline}
\begin{enumerate}
  \item Introduction and motivation (plasmonics, hot carriers, interband role).
  \item Theory (FGR, DOS, Boltzmann, non-equilibrium distributions, delta approximations).
  \item Intraband emission (equilibrium/non-equilibrium derivations; numeric vs. analytic; momentum-space validation).
  \item Interband emission (Au band parameters; anisotropic treatment; numeric implementation; results for X/L).
  \item Discussion (momentum conservation, phonon assistance, experiment comparison, limitations).
  \item Conclusions and outlook (summary; proposed next experiments/computations).
\end{enumerate}

\input{ai - section_interband.tex}

\appendix
\section{Derivation of the General Emission Integral (A1)}
Summing all transitions from $\ket{\mathbf{k}_{1}}$ to $\ket{\mathbf{k}_{2}}$ with spin degeneracy gives
\begin{align}
  \Gamma(\hbarw)&=2V\!\int\!\frac{d^{3}k_{1}}{(2\pi)^{3}}\,2V\!\int\!\frac{d^{3}k_{2}}{(2\pi)^{3}}\,\frac{2\pi}{\hbar}\,|\mu(\mathbf{k}_{1},\mathbf{k}_{2})|^{2}\,f(\mathbf{k}_{1})\bigl[1-f(\mathbf{k}_{2})\bigr]\delta\!\left(\E(\mathbf{k}_{2})-\E(\mathbf{k}_{1})+\hbarw\right),
\end{align}
recovering Eq.~\eqref{eq:general_rate}.

\section{Delta Approximation Details}
In 1D, total error $E_{N,\sigma}=E^{\mathrm{approx}}_{\sigma}+E^{\mathrm{numeric}}_{N,\sigma}$ with $E^{\mathrm{numeric}}_{N,\sigma}\propto N^{-2}$ (trapezoid rule) and $E^{\mathrm{approx}}_{\sigma}\to 0$ as $\sigma\to0$. In 2D integrals over $k_{1},k_{2}$ with $\delta(\E_{1}-\E_{2}-\hbarw)$, replacing the delta by $G_{\sigma}$ and integrating over a window $\pm m\sigma$ with $\Delta k=s\sigma$ preserves normalization and convergence.

\section{Photon Absorption and Momentum}
Free electrons cannot absorb photons because energy and momentum cannot be conserved simultaneously: $\mathbf{k}_{\mathrm{f}}-\mathbf{k}_{\mathrm{i}}=\mathbf{q}$ and $\E_{\mathrm{f}}-\E_{\mathrm{i}}=\hbar c|\mathbf{q}|$ cannot both hold with parabolic dispersion. Bloch electrons relax the momentum constraint through the lattice; intraband absorption remains effectively phonon-assisted.

\section{Non-equilibrium Distribution Notes}
The excitation term is $R(\omega_{\mathrm{L}})|E_{\mathrm{L}}|^{2}B(\E;\omega_{\mathrm{L}})$; $e$-$e$ relaxation yields $\Delta f=-\tau_{e\text{--}e}^{-1}\Delta f$, giving Eq.~\eqref{eq:nonthermal_dist}. Recombination is negligible for metals on the timescales considered.


\bibliographystyle{unsrt}
\bibliography{ai-references}

\end{document}
