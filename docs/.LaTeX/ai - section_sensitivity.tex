\section{Parameter Sensitivity and Physical Trends}
\subsection{Temperature dependence}
\begin{itemize}
  \item Higher $T$ broadens $\FD$, boosting low-$\hbarw$ emission and reducing sharpness near $\EF$.
  \item Analytic Eq.~\eqref{eq:analytic_eq} captures this via the Bose factor; energy-dependent $\Edos$ adds a mild hardening at larger $\hbarw$.
\end{itemize}

\subsection{Pump photon energy $\hbarwl$}
\begin{itemize}
  \item Non-equilibrium enhancement scales with $B(\E;\omega_{\mathrm{L}})$; as $\hbarwl$ grows, the terms in Eq.~\eqref{eq:B_kernel} separate and $\delta f$ grows until phase-space closes.
  \item For interband work, $\hbarwl$ must exceed the $5d\to6sp$ threshold; below threshold $\Gamma^{\nu}_{cv}$ vanishes.
\end{itemize}

\subsection{Pump intensity (through $\delta_{E}$)}
\begin{itemize}
  \item Small $\delta_{E}$: linear regime governed by $A(\hbarw)$; spectra shift modestly.
  \item Moderate $\delta_{E}$: quadratic corrections $B(\hbarw)\delta_{E}^{2}$ appear; clipping keeps occupations physical.
  \item Large $\delta_{E}$: would require adding recombination and $e$-$ph$ cooling; to be addressed if experimental conditions demand it.
\end{itemize}

\subsection{Effect of the density of states}
\begin{itemize}
  \item $\Edos(\E)\propto\sqrt{\E}$ enhances high-energy emission by weighting higher initial/final states.
  \item For anisotropic valleys, effective masses reshape $\Edos$; isotropization via Eq.~\eqref{eq:anisotropic_cov} provides an analytic route to compute the modified DOS.
\end{itemize}
